% !TEX TS-program = pdflatex
% !TEX encoding = UTF-8 Unicode

% This is a simple template for a LaTeX document using the "article" class.
% See "book", "report", "letter" for other types of document.

\documentclass[11pt]{article} % use larger type; default would be 10pt

\usepackage[utf8]{inputenc} % set input encoding (not needed with XeLaTeX)

%%% Examples of Article customizations
% These packages are optional, depending whether you want the features they provide.
% See the LaTeX Companion or other references for full information.

%%% PAGE DIMENSIONS
\usepackage{geometry} % to change the page dimensions
\geometry{letterpaper} % or letterpaper (US) or a5paper or....
\geometry{margin=0.5in} % for example, change the margins to 2 inches all round
% \geometry{landscape} % set up the page for landscape
%   read geometry.pdf for detailed page layout information

\usepackage{graphicx} % support the \includegraphics command and options

% \usepackage[parfill]{parskip} % Activate to begin paragraphs with an empty line rather than an indent

%%% PACKAGES
\usepackage{booktabs} % for much better looking tables
\usepackage{array} % for better arrays (eg matrices) in maths
\usepackage{paralist} % very flexible & customisable lists (eg. enumerate/itemize, etc.)
\usepackage{verbatim} % adds environment for commenting out blocks of text & for better verbatim
\usepackage{subfig} % make it possible to include more than one captioned figure/table in a single float
% These packages are all incorporated in the memoir class to one degree or another...
\usepackage{pgfplots}

%%% HEADERS & FOOTERS
\usepackage{fancyhdr} % This should be set AFTER setting up the page geometry
\pagestyle{fancy} % options: empty , plain , fancy
\renewcommand{\headrulewidth}{0pt} % customise the layout...
\lhead{}\chead{}\rhead{}
\lfoot{}\cfoot{\thepage}\rfoot{}

%%% SECTION TITLE APPEARANCE
\usepackage{sectsty}
\allsectionsfont{\sffamily\mdseries\upshape} % (See the fntguide.pdf for font help)
% (This matches ConTeXt defaults)

%%% ToC (table of contents) APPEARANCE
\usepackage[nottoc,notlof,notlot]{tocbibind} % Put the bibliography in the ToC
\usepackage[titles,subfigure]{tocloft} % Alter the style of the Table of Contents
\renewcommand{\cftsecfont}{\rmfamily\mdseries\upshape}
\renewcommand{\cftsecpagefont}{\rmfamily\mdseries\upshape} % No bold!

%%% END Article customizations

%%% The "real" document content comes below...

\title{CMPT 417 Project}
\author{Joel Teichroeb}
%\date{} % Activate to display a given date or no date (if empty),
         % otherwise the current date is printed 

\begin{document}
\maketitle

\section{Problem}
Investigating how different forms of CNF for generalized sudoku affect the speed in which different SAT solvers can show satisfiability.

\section{Emphasis}
My emphasis for this project is on using SAT solvers.

\section{Direct Solution}
I found a solver online\footnote{http://www.8325.org/sudoku/source.html} that was put into the public domain which almost supported generalized sudoku. I have done a few minor changes in order for it to read the same format that my other tools use, for example, being able to read a number larger than one digit. This solver is not the greatest direct solver that is available, but should represent a good baseline for how fast a direct solver will be.

\section{Specification-based Solution}
The specification being used is the one that has been provided on the course website. It is being run through the provided copy of Enfragmo. I have not included the copy of Enfragmo in the zip file as I do not want to distribute it anywhere. All my scripts assume that the provided binary is located in the users home directory. The input format is different than my tools, so instead of trying to make it somehow read the input format I want, I wrote a small program to take the input format I want and output the format the the specification expects.

\section{Benchmark Instance Families}
One family I use is randomly generated problems. There is a good generalized sudoku generator that I found online\footnote{http://ia.udl.cat/remository/Software/Sudoku-Generator/}. This generator has full control over how many empty spots there are on the sudoku board and I have found that it actually does generate interesting problems for the solvers to run. I wrote a set of scripts to run

\section{Emphasis On SAT}
\subsection{Methods}
Two methods of generating CNF formulas were generated for this problem. The first method is very simple. It uses 1 variable for each possible number on each square of the map. For a 16x16 sudoku this means it has a total of 4096 variables. The clauses for this one are for the most part very simple, consisting of only two variables each for most of the clauses. The second method is a lot more complicated. The idea for it came from the paper\footnote{http://www.cs.sfu.ca/~mitchell/cmpt-417/2013-Spring/Readings/2-Malik-Zhang-CACM-SAT-Review.pdf} we were provided on solving graph colouring using binary. I use only four variables for each square on the board which represent the number in binary. For example if we have variables $a = 1, b = 0, c = 1, d = 0$ the value of the square would be $5$. This method works very well for 16x16 as no bits are wasted, unlike for a 9x9 puzzle as you would still need 4 bits to represent it. This has a huge advantage in the number of variables used for the problem as each square only needs 4, unlike my first method where they need 16, meaning the total number of variables for a 16x16 is only 1024. But, this comes at a cost in the formula that is generated as the clauses now all use at least eight variables, making it a lot harder to eliminate clauses from the formula.
\subsection{Solvers}
There are many SAT solvers out that that have been released as open source.
Oddly enough, a large number of them seem to be based on MiniSat.
I chose four SAT solvers that have done generally very well in past SAT competitions, which were also easy to build on my machine.
These are Glucose\footnote{https://www.lri.fr/~simon/?page=glucose}, MiniSat\footnote{http://minisat.se/}, Lingeling\footnote{http://fmv.jku.at/lingeling/}, and Clasp\footnote{http://www.cs.uni-potsdam.de/clasp/}.

\section{Results}
\subsection{First Family}
The first benchmark family has a simple architecture, but provides very interesting data through running the solvers many times. Simply it generates a 16x16 sudoku puzzle randomly with N holes. N starts at 1 and increases by 5 up to 251. Each value of N is run 10 times. Solutions are not guaranteed to be unique, but they are generally not the easiest problems to solve. I set a timeout of 10 seconds as I thought that would be a reasonable limit.

\subsubsection{Specification}
\begin{tikzpicture}
\begin{axis}[width=550pt]
\addplot+[mark=none] table [col sep=comma,trim cells=true,y=spec] {data.csv};
\addlegendentry{Specification Runtime}
\end{axis}
\end{tikzpicture}

Overall the specification is fairly consistent in it's runtime. The more holes that are in the puzzle, the harder it is for it to find a solution. The odd thing about this data is that for very small inputs it still has a runtime of at least one second. This suggests some overhead somewhere in the system, but it is impossible for me to find where that is without access to the Enfragmo source code.

\subsubsection{Direct}
\begin{tikzpicture}
\begin{axis}[width=550pt]
\addplot+[mark=none] table [col sep=comma,trim cells=true,y=direct] {data.csv};
\addlegendentry{Direct Runtime}
\end{axis}
\end{tikzpicture}

The direct method provides runtime that could be reasonably expected from a random solver off the internet. When the problem is easy the direct solution provides a consistently fast answer, but once it starts getting just a little harder it quickly leads to timeouts and makes the direction solution completely impractical.

\subsubsection{SAT First Method}
\begin{tikzpicture}
\begin{axis}[width=540pt]
\addplot+[mark=none] table [col sep=comma,trim cells=true,y=mini1] {data.csv};
\addlegendentry{MiniSat}
\addplot+[mark=none] table [col sep=comma,trim cells=true,y=glu1] {data.csv};
\addlegendentry{Glucose}
\addplot+[mark=none] table [col sep=comma,trim cells=true,y=ling1] {data.csv};
\addlegendentry{Lingeling}
\addplot+[mark=none] table [col sep=comma,trim cells=true,y=clasp1] {data.csv};
\addlegendentry{Clasp}
\end{axis}
\end{tikzpicture}

This SAT method works very well on almost all of the solvers. Clasp, unfortunately does very poorly compared to the other SAT solvers as it seems to follow an exponential curve the more holes that are added. Lingeling has a fairly consistent time just a bit slower than MiniSat and Glucose. I would guess this slower speed is the result of a slower parser.

\subsubsection{SAT Second Method}
\begin{tikzpicture}
\begin{axis}[width=540pt]
\addplot+[mark=none] table [col sep=comma,trim cells=true,y=mini2] {data.csv};
\addlegendentry{MiniSat}
\addplot+[mark=none] table [col sep=comma,trim cells=true,y=glu2] {data.csv};
\addlegendentry{Glucose}
\addplot+[mark=none] table [col sep=comma,trim cells=true,y=ling2] {data.csv};
\addlegendentry{Lingeling}
\addplot+[mark=none] table [col sep=comma,trim cells=true,y=clasp2] {data.csv};
\addlegendentry{Clasp}
\end{axis}
\end{tikzpicture}

This SAT method is terrible compared to the first one. Surprisingly Lingeling actually does the worst with this method instead of Clasp, like in the last method. MiniSat and Glucose still do the best early on. The solvers actually get slow later than the Direct Solver did, meaning that it's better to use this worse SAT method than it is to use a badly made direct solver.

\newpage

\subsection{Second Family}

The second family of benchmarks I am using are 5 simple sudoku boards. The first is simply an empty board. For a human, it is trivial to prove that this is solvable. The second is a board with just one number entered. This is very similar to the empty case, so results should be similar. The third one is a simple unsat test. It simply has two 1's on the same row. It should be trivial to show that this board is not satisfiable. The fourth is a slightly more complex unsat case. In this one the entire board is filled, but one number is removed. The number that has been removed gets moved somewhere else in the row instead. It is harder for a human to see this one as there are so many numbers, but should still be trivial. The last case is an already solved board which should also be trivial to show.

\begin{table}[h]
    \begin{tabular}{|l|l|l|l|l|l|l|l|l|l|l|}
    \hline
    ~      & Mini 1 & Mini 2 & Glu 1 & Glu 2 & Lini 1 & Ling 2 & Clasp 1 & Clasp 2 & Direct & Specification \\ \hline
    Empty  & 0.05          & 10.00          & 0.05          & 10.00          & 0.08            & 10.00            & 0.47        & 10.00        & 1.92   & 1.83          \\ \hline
    Single & 0.05          & 10.00          & 0.05          & 10.00          & 0.08            & 10.00            & 0.47        & 10.00        & 1.94   & 1.88          \\ \hline
    Unsat1 & 0.04          & 0.07           & 0.04          & 0.07           & 0.07            & 0.17             & 0.06        & 0.09         & 0.00   & 1.24          \\ \hline
    Unsat2 & 0.04          & 0.07           & 0.04          & 0.08           & 0.08            & 0.17             & 0.07        & 0.09         & 0.00   & 1.15          \\ \hline
    Solved & 0.05          & 0.08           & 0.05          & 0.08           & 0.08            & 0.19             & 0.07        & 0.10         & 0.00   & 1.19          \\ \hline
    \end{tabular}
\end{table}

The results here make sense for the most part. The only part that surprised me is how the second SAT method timed out in the almost empty cases. The direct and specification also took longer than I would have thought. The direct approach is the best one at determining an already solved board or a trivially unsat board.

\section{Conclusions}

Some methods of generating CNF formulas are better than others in almost every case and not all SAT solvers are created equally. Using SAT to solve problems is almost surely a good bet as long as one writes good CNF and uses a world renowned SAT solver.

\end{document}
