% !TEX TS-program = pdflatex
% !TEX encoding = UTF-8 Unicode

% This is a simple template for a LaTeX document using the "article" class.
% See "book", "report", "letter" for other types of document.

\documentclass[11pt]{article} % use larger type; default would be 10pt

\usepackage[utf8]{inputenc} % set input encoding (not needed with XeLaTeX)

%%% Examples of Article customizations
% These packages are optional, depending whether you want the features they provide.
% See the LaTeX Companion or other references for full information.

%%% PAGE DIMENSIONS
\usepackage{geometry} % to change the page dimensions
\geometry{letterpaper} % or letterpaper (US) or a5paper or....
\geometry{margin=1in} % for example, change the margins to 2 inches all round
% \geometry{landscape} % set up the page for landscape
%   read geometry.pdf for detailed page layout information

\usepackage{graphicx} % support the \includegraphics command and options

% \usepackage[parfill]{parskip} % Activate to begin paragraphs with an empty line rather than an indent

%%% PACKAGES
\usepackage{booktabs} % for much better looking tables
\usepackage{array} % for better arrays (eg matrices) in maths
\usepackage{paralist} % very flexible & customisable lists (eg. enumerate/itemize, etc.)
\usepackage{verbatim} % adds environment for commenting out blocks of text & for better verbatim
\usepackage{subfig} % make it possible to include more than one captioned figure/table in a single float
% These packages are all incorporated in the memoir class to one degree or another...

%%% HEADERS & FOOTERS
\usepackage{fancyhdr} % This should be set AFTER setting up the page geometry
\pagestyle{fancy} % options: empty , plain , fancy
\renewcommand{\headrulewidth}{0pt} % customise the layout...
\lhead{}\chead{}\rhead{}
\lfoot{}\cfoot{\thepage}\rfoot{}

%%% SECTION TITLE APPEARANCE
\usepackage{sectsty}
\allsectionsfont{\sffamily\mdseries\upshape} % (See the fntguide.pdf for font help)
% (This matches ConTeXt defaults)

%%% ToC (table of contents) APPEARANCE
\usepackage[nottoc,notlof,notlot]{tocbibind} % Put the bibliography in the ToC
\usepackage[titles,subfigure]{tocloft} % Alter the style of the Table of Contents
\renewcommand{\cftsecfont}{\rmfamily\mdseries\upshape}
\renewcommand{\cftsecpagefont}{\rmfamily\mdseries\upshape} % No bold!

%%% END Article customizations

%%% The "real" document content comes below...

\title{CMPT 417 Project}
\author{Joel Teichroeb}
%\date{} % Activate to display a given date or no date (if empty),
         % otherwise the current date is printed 

\begin{document}
\maketitle

\section{Problem}
Investigating how different forms of CNF for generalized sudoku affect the speed in which different SAT solvers can show satisfiability or unsatisfiability.

\section{Emphasis}
My emphasis for this project is on using SAT solvers.

\section{Direct Solution}
I found a solver online that was put into the public domain which almost supported generalized sudoku. I have done a few minor changes in order for it to read the same format that my other tools use, for example, being able to read a number larger than one digit. One trouble I have been having though is that this solver is not very fast in a lot of situations, but I have not been able to find any other solvers online for generalized sudoku.

\section{Specification-based Solution}
I am just using the specification provided on the course website, running it through the provided copy of Enfragmo. The input format is different than my tools, so instead of trying to make it somehow read the input format I want, I wrote a little program to take the input format I want and output the format the the specification expects.

\section{Benchmark Instance Families}
I've been working on a random generalized sudoku generator to use for one family of benchmarks. I have tried to find a generalized sudoku generator online to save me effort in making this part, but none of the generators are easily converted to the generalized case.

\section{Emphasis Direction}
Using a number of different SAT solvers, including but not limited to MiniSat, Glucose, and SatELite on multiple different ways to represent a generalized sudoku in CNF. Right now the biggest obstacle I have is actually writing the code for more ways of representing the problem in CNF, but this should not be too hard in the end.

\end{document}
